\usepackage[utf8]{inputenc}
% \usepackage[T1]{fontenc} %% puvodni evropske fonty->spatne umistene hacky nad c, d s hackem ma za sebou mezeru
\usepackage[IL2]{fontenc} %fonty vyladene pro cestinu/slovenstinu
\usepackage[czech,english]{babel}
\usepackage[table]{xcolor}
\usepackage{mathptmx}  %% Adobe Times Roman
\usepackage{amsmath,amssymb,amsthm}
\usepackage[Tisk]{slim.thesis}

\usepackage{amsmath}
\usepackage{amsfonts}
\usepackage{amsthm}
\usepackage{mathrsfs} % \psaci font
\usepackage{amssymb}
\usepackage[all]{xy}
\usepackage{multirow}
\usepackage{floatrow}
\floatsetup[table]{capposition=top}
\usepackage[sort&compress]{natbib}
\usepackage{tocloft}
\usepackage{acronym}
\usepackage{lipsum}
\usepackage{listings}
\usepackage{xspace}

% \usepackage{acro}
\usepackage{relsize}
\usepackage[nointegrals]{wasysym}

%%  fonty
\let\tluste\mathbb
\let\kroucene\mathcal
\let\psaci\mathscr

\definecolor{izicolor}{RGB}{191,0,0}


%%%%%%%%%%%%%%%%%%%%%%%%%%%%%%%%%%%%%%%%%%%%%%%%%%%%%%%%%%%%%%%%%%%%%%%%%%%%%%%%
%%%%%%%%%%%%%%%%%%%%%%%%%%%%%%%%%%%%%%%%%%%%%%%%%%%%%%%%%%%%%%%%%%%%%%%%%%%%%%%%
\makeatletter
	\def\th@plain{\slshape}	% zneni tvrzeni chci v \sl, ne v \it
	\def\thmhead#1#2#3{%
		\thmname {#1}\thmnumber{\@ifnotempty{#1}{ }\@upn{#2}}\thmnote{ {\the\thm@notefont #3}}%
	}
	\newcommand{\citeparagraph}[2]{%
		\@ifempty{#1}{\cite{#2}}{\cite[#1]{#2}}%
	}
\makeatother

\newcommand{\iziprintcounter}[1]{%
	{\rm({\it\roman{#1}}\/)}%
}
\newcommand{\iref}[1]{%
	{\rm({\it\ref{#1}}\/)}%
}

\newenvironment{enumerati}{\begin{enumerate}[\rm(\it i\rm\/)]}{\end{enumerate}}

\newcounter{izitmpcounter}
\newcommand{\dklabel}[1]{%
	\setcounter{izitmpcounter}{#1}%
	\iziprintcounter{izitmpcounter}%
}

	%  pekne recke pismena
\let\izitmp\epsilon
\let\epsilon\varepsilon
\let\varepsilon\izitmp
\let\izitmp\phi
\let\phi\varphi
\let\varphi\izitmp
\let\setminus\smallsetminus


\DeclareSymbolFont{izilettersA}{U}{pxmia}{m}{it}
\DeclareMathSymbol{\betaup}{\mathord}{izilettersA}{12}

% \newcommand{\C}{\tluste{C}}
\newcommand{\R}{\tluste{R}}
\newcommand{\Q}{\tluste{Q}}
% \newcommand{\Z}{\tluste{Z}}
\newcommand{\N}{\tluste{N}}


%norma a absolutni hodnota
\newcommand{\norm}[1]{\left\lVert{#1}\right\rVert}
\newcommand{\abs}[1]{\left\lvert{#1}\right\rvert}
\newcommand{\latex}{\LaTeX\xspace}
%sipka pro obsah .fasta souborů
\newcommand{\sipka}{\begin{tt}\symbol{'76}\end{tt}}

\definecolor{Gray}{gray}{0.9}
\definecolor{Green}{RGB}{0,100,25}

% \usepackage{fancyvrb}
% \fvset{tabsize=2}

\lstdefinestyle{customc}{
  breaklines=true,
  frame=l,
  language=python,
  basicstyle=\footnotesize\rmfamily,
  keywordstyle=\bfseries\color{green!40!black},
  commentstyle=\itshape\color{purple!40!black},
  stringstyle=\color{orange},
  tabsize=2,
  showstringspaces=false,
}

\input{glyphtounicode}
\pdfglyphtounicode{visiblespace}{A0}
\pdfglyphtounicode{blank}{A0}
\pdfglyphtounicode{visualspace}{A0}
\pdfglyphtounicode{uni2423}{A0}
\pdfgentounicode=1
\def\@xobeysp{\textcolor{white}{\char32}}

\lstset{ %
  style=customc
%   backgroundcolor=\color{white},   % choose the background color; you must add \usepackage{color} or \usepackage{xcolor}
%   basicstyle=\footnotesize,        % the size of the fonts that are used for the code
%   breakatwhitespace=false,         % sets if automatic breaks should only happen at whitespace
%   breaklines=true,                 % sets automatic line breaking
%   captionpos=t,                    % sets the caption-position to bottom
%   commentstyle=\color{Green},    % comment style
% %   deletekeywords={...},            % if you want to delete keywords from the given language
% %   escapeinside={\%*}{*)},          % if you want to add LaTeX within your code
% %   extendedchars=true,              % lets you use non-ASCII characters; for 8-bits encodings only, does not work with UTF-8
% %   frame=none,                    % adds a frame around the code
% %   keepspaces=true,                 % keeps spaces in text, useful for keeping indentation of code (possibly needs columns=flexible)
%   keywordstyle=\color{blue},       % keyword style
%   language=python,                 % the language of the code
% %   morekeywords={*,...},            % if you want to add more keywords to the set
% %   numbers=none,                    % where to put the line-numbers; possible values are (none, left, right)
% %   numbersep=5pt,                   % how far the line-numbers are from the code
% %   numberstyle=\tiny\color{black}, % the style that is used for the line-numbers
% %   rulecolor=\color{black},         % if not set, the frame-color may be changed on line-breaks within not-black text (e.g. comments (green here))
% %   showspaces=false,                % show spaces everywhere adding particular underscores; it overrides 'showstringspaces'
% %   showstringspaces=false,          % underline spaces within strings only
%   showtabs=false,                  % show tabs within strings adding particular underscores
% %   stepnumber=2,                    % the step between two line-numbers. If it's 1, each line will be numbered
%   stringstyle=\color{red},     % string literal style
%   tabsize=2,                       % sets default tabsize to 2 spaces
%   title=\lstname                   % show the filename of files included with \lstinputlisting; also try caption instead of title
}

\renewcommand{\cftchappresnum}{Chapter }
